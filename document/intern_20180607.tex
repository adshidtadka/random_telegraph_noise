\documentclass{jsarticle}
\usepackage[dvipdfmx]{graphicx}
\usepackage{float}
\usepackage{amsmath}
\usepackage[dvipdfmx]{color}
\usepackage{listings}
\begin{document}
\title{小野寺研インターン}
\author{澤 孝晃}
\maketitle

\section{序論}

今回のインターンでは、微細デバイスに発生するランダムテレグラフノイズ(RTN)をリング発振回路を用いて測定し、RTNが回路性能の最悪分布に与える影響を評価する。測定対象であるRTNは統計的な性質を持っており、各種統計的な性質をモデル化することが目的である。統計的な評価を行うために、同じ寸法の大量のデバイスの電流特性の時間変化を測定し、デバイス毎に観測される電流値変動の振幅および捕獲・放出するまでの平均時間などを測定する。

\section{方法}

今回の測定環境では、FPGAボードとPCを使って、スロット0からスロット71のリングオシレータ(RO)を、セクション0からセクション383まで384個のセクションの発信周波数を測定する。各セクションそれぞれ10秒ずつ1msの間隔で測定するため、1つのリングオシレータに対して1時間程度かかる。

\section{最大周波数に関する結果}






\end{document}
